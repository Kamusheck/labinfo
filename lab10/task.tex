\documentclass[a4paper, 16pt]{article}
\usepackage[english, russian]{babel}
\usepackage[T2A]{fontenc}
\usepackage[utf8]{inputenc}
\usepackage{amssymb}
\usepackage{amsmath}

\begin{document}
\thispagestyle{empty}
\noindentций существуют пределы в точке $x_0$, принадлежащей в данном случае их множеству задания, что и означает их непрерывность в этой точке. Иначе говоря, утверждение следствия 2 утверждения $6^0$ является просто частным слу-
чаем этого утверждения, когда точка, в которой рассматри- вается предел, приналежит области задания функцийций.

\textbf{5.11. Бесконечно малые и бесконечно большие функции}
Все рассматриваемые в этом пункте функции будем предпо- лагать определенными на множестве $X \subset \textbf{R}$ и рассматривать их конечные и бесконечные пределы при стремлении аргу- мента к конечной или к бесконечно удаленной точке $х_0$.
\textbf{Определение 12.}\textit{Функция $\alpha$:$X \rightnarrow \textbf{R} $ называется бесконечно малой при $x \rightnarrow x_0$,если}
\[
\lim_{\substack{x \rightarrow x_0} $\alpha$(x)=0.
\eqno\textbf(5.45)\]
  
Бесконечно малые функции играют особую роль среди всех функций, имеющих предел, связанную, в частности, с тем, что общее понятие конечного предела может быть све- дено к понятию бесконечно малой. Сформулируем это ут- верждение в виде леммы.
\textbf{Л Е М М А 6.} \textit{Конечный предел $\lim_{x \rightarrow x_0} f(x)$ существует и равен а тогда и только тогда, когда $f(x) = \textit{a} +\alpha(x),  x \in \boldsymbol{X}$, где $\alpha = \alpha(x)$- бесконечно малая при $x \rightnarrow x_0$.} 
  

  
