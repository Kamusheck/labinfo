\documentclass[a4paper, 20pt]{article}
\usepackage[english, russian]{babel}
\usepackage[T2A]{fontenc}
\usepackage[utf8x]{inputenc}
\usepackage{amssymb}
\usepackage{amsmath}
\usepackage{indentfirst}


\begin{document}
\begin{large}
\thispagestyle{empty}
\noindentций существуют пределы в точке $x_0$, принадлежащей в данном случае их множеству задания, что и означает их непрерывность в этой точке. Иначе говоря, утверждение следствия 2 утверждения $6^0$ является просто частным случаем этого утверждения, когда точка, в которой рассматривается предел, приналежит области задания функций.
\\


\noindent\textbf{{5.11. Бесконечно малые}}

\noindent\textbf{{и бесконечно большие функции}}
\\

\noindentВсе рассматриваемые в этом пункте функции будем предполагать определенными на множестве $X \subset \textbf{R}$ и рассматривать их конечные и бесконечные пределы при стремлении аргу-мента к конечной или к бесконечно удаленной точке $x_0$.

\noindent\textbf{Определение 12.}\textit{Функция $\alpha$: $X \rightarrow \textbf{R} $ называется бесконечно малой при $x \rightarrow x_0$, если}
\[
\lim_{x \rightarrow x_0} \alpha(x)=0.
 \eqno\textbf(5.45)\]

 Бесконечно малые функции играют особую роль среди всех функций, имеющих предел, связанную, в частности, с тем, что общее понятие конечного предела может быть сведено к понятию бесконечно малой. Сформулируем это утверждение в виде леммы.

\noindent\textbf{Л Е М М А 6.} \textit{Конечный предел $\lim\limits_{x \to \x_0} f(x)$ существует и равен \textit{а} тогда и только тогда, когда $f(x) = \textit{a} +\alpha(x),  x \in \boldsymbol{X}$, где $\alpha = \alpha(x)$- бесконечно малая при $x \rightarrow x_0$.}
\\\textsc{Доказательство}. Если $\lim\limits_{x \rightarrow x_0} f(x)=\textit{a}$, то, положив $\alpha(x)=f(x)- \textit{a},  x \in\boldsymbol{X}$, получим, что
\[
\lim_{x \rightarrow x_0} \alpha(x)= \lim_{x \rightarrow x_0} f(x)-\textit{a=a}-\extit{a=0}}.
\]
Наоборот, если $f(x)=\textit{a}+\alpha(x),  x \in \boldsymbol{X}$ и $\lim\limits_{x \rightarrow x_0} \alpha(x)=0$, то
\[
\lim_{x \rightarrow x_0} f(x)=\textit{a}+ \lim_{x \rightarrow x_0} \alpha(x)=\textit{a}. \square
\]
\noindent\textbf{ТЕОРЕМА 3.}\textit{Сумма и произведение конечного числа бесконечно малыx при  $x \rightarrow x_0$, а также и произведение бесконеч-} 

\[
\\
\centering\overline{\textit{194}}
\]

\end{large}
\end{document}

  

  
